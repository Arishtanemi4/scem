\documentclass[]{article}
\usepackage{lmodern}
\usepackage{amssymb,amsmath}
\usepackage{ifxetex,ifluatex}
\usepackage{fixltx2e} % provides \textsubscript
\ifnum 0\ifxetex 1\fi\ifluatex 1\fi=0 % if pdftex
  \usepackage[T1]{fontenc}
  \usepackage[utf8]{inputenc}
\else % if luatex or xelatex
  \ifxetex
    \usepackage{mathspec}
  \else
    \usepackage{fontspec}
  \fi
  \defaultfontfeatures{Ligatures=TeX,Scale=MatchLowercase}
\fi
% use upquote if available, for straight quotes in verbatim environments
\IfFileExists{upquote.sty}{\usepackage{upquote}}{}
% use microtype if available
\IfFileExists{microtype.sty}{%
\usepackage{microtype}
\UseMicrotypeSet[protrusion]{basicmath} % disable protrusion for tt fonts
}{}
\usepackage[margin=1in]{geometry}
\usepackage{hyperref}
\hypersetup{unicode=true,
            pdftitle={Assignment 03 - Solutions},
            pdfauthor={YOUR\_NAME (YOUR\_STUDENT\_ID)},
            pdfborder={0 0 0},
            breaklinks=true}
\urlstyle{same}  % don't use monospace font for urls
\usepackage{color}
\usepackage{fancyvrb}
\newcommand{\VerbBar}{|}
\newcommand{\VERB}{\Verb[commandchars=\\\{\}]}
\DefineVerbatimEnvironment{Highlighting}{Verbatim}{commandchars=\\\{\}}
% Add ',fontsize=\small' for more characters per line
\usepackage{framed}
\definecolor{shadecolor}{RGB}{248,248,248}
\newenvironment{Shaded}{\begin{snugshade}}{\end{snugshade}}
\newcommand{\AlertTok}[1]{\textcolor[rgb]{0.94,0.16,0.16}{#1}}
\newcommand{\AnnotationTok}[1]{\textcolor[rgb]{0.56,0.35,0.01}{\textbf{\textit{#1}}}}
\newcommand{\AttributeTok}[1]{\textcolor[rgb]{0.77,0.63,0.00}{#1}}
\newcommand{\BaseNTok}[1]{\textcolor[rgb]{0.00,0.00,0.81}{#1}}
\newcommand{\BuiltInTok}[1]{#1}
\newcommand{\CharTok}[1]{\textcolor[rgb]{0.31,0.60,0.02}{#1}}
\newcommand{\CommentTok}[1]{\textcolor[rgb]{0.56,0.35,0.01}{\textit{#1}}}
\newcommand{\CommentVarTok}[1]{\textcolor[rgb]{0.56,0.35,0.01}{\textbf{\textit{#1}}}}
\newcommand{\ConstantTok}[1]{\textcolor[rgb]{0.00,0.00,0.00}{#1}}
\newcommand{\ControlFlowTok}[1]{\textcolor[rgb]{0.13,0.29,0.53}{\textbf{#1}}}
\newcommand{\DataTypeTok}[1]{\textcolor[rgb]{0.13,0.29,0.53}{#1}}
\newcommand{\DecValTok}[1]{\textcolor[rgb]{0.00,0.00,0.81}{#1}}
\newcommand{\DocumentationTok}[1]{\textcolor[rgb]{0.56,0.35,0.01}{\textbf{\textit{#1}}}}
\newcommand{\ErrorTok}[1]{\textcolor[rgb]{0.64,0.00,0.00}{\textbf{#1}}}
\newcommand{\ExtensionTok}[1]{#1}
\newcommand{\FloatTok}[1]{\textcolor[rgb]{0.00,0.00,0.81}{#1}}
\newcommand{\FunctionTok}[1]{\textcolor[rgb]{0.00,0.00,0.00}{#1}}
\newcommand{\ImportTok}[1]{#1}
\newcommand{\InformationTok}[1]{\textcolor[rgb]{0.56,0.35,0.01}{\textbf{\textit{#1}}}}
\newcommand{\KeywordTok}[1]{\textcolor[rgb]{0.13,0.29,0.53}{\textbf{#1}}}
\newcommand{\NormalTok}[1]{#1}
\newcommand{\OperatorTok}[1]{\textcolor[rgb]{0.81,0.36,0.00}{\textbf{#1}}}
\newcommand{\OtherTok}[1]{\textcolor[rgb]{0.56,0.35,0.01}{#1}}
\newcommand{\PreprocessorTok}[1]{\textcolor[rgb]{0.56,0.35,0.01}{\textit{#1}}}
\newcommand{\RegionMarkerTok}[1]{#1}
\newcommand{\SpecialCharTok}[1]{\textcolor[rgb]{0.00,0.00,0.00}{#1}}
\newcommand{\SpecialStringTok}[1]{\textcolor[rgb]{0.31,0.60,0.02}{#1}}
\newcommand{\StringTok}[1]{\textcolor[rgb]{0.31,0.60,0.02}{#1}}
\newcommand{\VariableTok}[1]{\textcolor[rgb]{0.00,0.00,0.00}{#1}}
\newcommand{\VerbatimStringTok}[1]{\textcolor[rgb]{0.31,0.60,0.02}{#1}}
\newcommand{\WarningTok}[1]{\textcolor[rgb]{0.56,0.35,0.01}{\textbf{\textit{#1}}}}
\usepackage{graphicx,grffile}
\makeatletter
\def\maxwidth{\ifdim\Gin@nat@width>\linewidth\linewidth\else\Gin@nat@width\fi}
\def\maxheight{\ifdim\Gin@nat@height>\textheight\textheight\else\Gin@nat@height\fi}
\makeatother
% Scale images if necessary, so that they will not overflow the page
% margins by default, and it is still possible to overwrite the defaults
% using explicit options in \includegraphics[width, height, ...]{}
\setkeys{Gin}{width=\maxwidth,height=\maxheight,keepaspectratio}
\IfFileExists{parskip.sty}{%
\usepackage{parskip}
}{% else
\setlength{\parindent}{0pt}
\setlength{\parskip}{6pt plus 2pt minus 1pt}
}
\setlength{\emergencystretch}{3em}  % prevent overfull lines
\providecommand{\tightlist}{%
  \setlength{\itemsep}{0pt}\setlength{\parskip}{0pt}}
\setcounter{secnumdepth}{0}
% Redefines (sub)paragraphs to behave more like sections
\ifx\paragraph\undefined\else
\let\oldparagraph\paragraph
\renewcommand{\paragraph}[1]{\oldparagraph{#1}\mbox{}}
\fi
\ifx\subparagraph\undefined\else
\let\oldsubparagraph\subparagraph
\renewcommand{\subparagraph}[1]{\oldsubparagraph{#1}\mbox{}}
\fi

%%% Use protect on footnotes to avoid problems with footnotes in titles
\let\rmarkdownfootnote\footnote%
\def\footnote{\protect\rmarkdownfootnote}

%%% Change title format to be more compact
\usepackage{titling}

% Create subtitle command for use in maketitle
\providecommand{\subtitle}[1]{
  \posttitle{
    \begin{center}\large#1\end{center}
    }
}

\setlength{\droptitle}{-2em}

  \title{Assignment 03 - Solutions}
    \pretitle{\vspace{\droptitle}\centering\huge}
  \posttitle{\par}
  \subtitle{Statistical Computing and Empirical Methods}
  \author{YOUR\_NAME (YOUR\_STUDENT\_ID)}
    \preauthor{\centering\large\emph}
  \postauthor{\par}
    \date{}
    \predate{}\postdate{}
  

\begin{document}
\maketitle

\hypertarget{a-word-of-advice}{%
\subsection{A word of advice}\label{a-word-of-advice}}

Think of the SCEM labs as going to the gym: if you pay a gym membership,
but instead of working out you use a machine to lift the weights for
you, you won't get the benefits.

ChatGPT, DeepSeek, Claude and other GenAI tools can provide answers to
most of the questions below. Before you try that, please consider the
following: answering the specific questions below is not the point of
this assignment. Instead, the questions are designed to give you the
chance to develop a better understanding of estimation concepts and a
certain level of \emph{\textbf{statistical thinking}}. These are
essential skills for any data scientist, even if they end up using
generative AI - to write an effective prompt and to catch the common
(often subtle) errors that AI produces when trying to solve anything
non-trivial.

A very important part of this learning involves not having the answers
ready-made for you, but instead taking the time to actually search for
the answer, trying something, getting it wrong, and trying again.

So, make the best use of this session. The assignments are not marked,
so it is much better to try the yourself even if you get incorrect
answers (you'll be able to correct yourself later when you receive
feedback) than to submit a perfect, but GPT'd solution.

\begin{center}\rule{0.5\linewidth}{0.5pt}\end{center}

\hypertarget{important-notes}{%
\subsection{IMPORTANT NOTES:}\label{important-notes}}

\begin{itemize}
\tightlist
\item
  \textbf{DO NOT} change the code block names. Enter your solutions to
  each question into the predefined code blocks.
\item
  \textbf{DO NOT} add calls to \texttt{install.packages()} into your
  solutions. Some questions may require you to load packages using
  \texttt{library()}. Please do not use any other packages except the
  ones explicitly listed in the ``setup'' code block below.
\end{itemize}

\begin{center}\rule{0.5\linewidth}{0.5pt}\end{center}

\hypertarget{setup}{%
\subsection{Setup}\label{setup}}

This code block below sets up your session. The only think you should
change in it is to replace \texttt{ABCDEF} by your student ID number,
which will be used as the seed for your random number generators.

\hypertarget{part-i-point-estimation-of-parameters}{%
\subsection{Part I: Point estimation of
parameters}\label{part-i-point-estimation-of-parameters}}

\hypertarget{q1.-simulating-data-and-estimating-the-mean}{%
\subsubsection{Q1. Simulating data and estimating the
mean}\label{q1.-simulating-data-and-estimating-the-mean}}

\begin{Shaded}
\begin{Highlighting}[]
\DocumentationTok{\#\# Question 1.a}
\FunctionTok{set.seed}\NormalTok{(MY\_STUDENT\_ID) }\CommentTok{\# \textless{}{-}{-}{-} Don\textquotesingle{}t change this}

\CommentTok{\# Your code here}
\CommentTok{\# ...}
\CommentTok{\# ...}
\CommentTok{\# x.mean \textless{}{-} ...}
\CommentTok{\# x.var  \textless{}{-} ...}
\end{Highlighting}
\end{Shaded}

\begin{Shaded}
\begin{Highlighting}[]
\DocumentationTok{\#\# Question 1.b}
\FunctionTok{set.seed}\NormalTok{(MY\_STUDENT\_ID) }\CommentTok{\# \textless{}{-}{-}{-} Don\textquotesingle{}t change this}

\CommentTok{\# Your code here}
\DocumentationTok{\#\#}
\end{Highlighting}
\end{Shaded}

\begin{center}\rule{0.5\linewidth}{0.5pt}\end{center}

\hypertarget{q2.-sampling-distribution-of-the-mean}{%
\subsubsection{Q2. Sampling distribution of the
mean}\label{q2.-sampling-distribution-of-the-mean}}

\begin{Shaded}
\begin{Highlighting}[]
\DocumentationTok{\#\# Question 2.a}
\FunctionTok{set.seed}\NormalTok{(MY\_STUDENT\_ID) }\CommentTok{\# \textless{}{-}{-}{-} Don\textquotesingle{}t change this}

\CommentTok{\# Your code here}
\DocumentationTok{\#\#}
\end{Highlighting}
\end{Shaded}

\begin{Shaded}
\begin{Highlighting}[]
\DocumentationTok{\#\# Question 2.b}
\FunctionTok{set.seed}\NormalTok{(MY\_STUDENT\_ID) }\CommentTok{\# \textless{}{-}{-}{-} Don\textquotesingle{}t change this}

\CommentTok{\# Your code here}
\DocumentationTok{\#\#}
\end{Highlighting}
\end{Shaded}

\begin{center}\rule{0.5\linewidth}{0.5pt}\end{center}

\hypertarget{q3.-bias-of-an-estimator}{%
\subsubsection{Q3. Bias of an
estimator}\label{q3.-bias-of-an-estimator}}

\begin{Shaded}
\begin{Highlighting}[]
\DocumentationTok{\#\# Question 3}
\FunctionTok{set.seed}\NormalTok{(MY\_STUDENT\_ID) }\CommentTok{\# \textless{}{-}{-}{-} Don\textquotesingle{}t change this}

\CommentTok{\# Your code here}
\DocumentationTok{\#\#}
\end{Highlighting}
\end{Shaded}

\begin{center}\rule{0.5\linewidth}{0.5pt}\end{center}

\hypertarget{discussion-maximum-likelihood-estimation-mle}{%
\subsubsection{Discussion: Maximum Likelihood Estimation
(MLE)}\label{discussion-maximum-likelihood-estimation-mle}}

Example of using the R function \texttt{optim()} to estimate the MLE
values for the mean and variance of a normal distribution. \emph{You
don't need to change anything in the code block below.}

\begin{Shaded}
\begin{Highlighting}[]
\DocumentationTok{\#\# This is just an example}

\FunctionTok{set.seed}\NormalTok{(}\DecValTok{1}\NormalTok{)}
\NormalTok{n    }\OtherTok{\textless{}{-}} \DecValTok{100000} \CommentTok{\# sample size}
\NormalTok{mu   }\OtherTok{\textless{}{-}} \DecValTok{12}     \CommentTok{\# true mean}
\NormalTok{var  }\OtherTok{\textless{}{-}} \DecValTok{9}      \CommentTok{\# true variance}


\CommentTok{\# Generate sample}
\NormalTok{x }\OtherTok{\textless{}{-}} \FunctionTok{rnorm}\NormalTok{(n, }\AttributeTok{mean =}\NormalTok{ mu, }\AttributeTok{sd =} \FunctionTok{sqrt}\NormalTok{(var))}

\CommentTok{\# log{-}likelihood for Normal(mu, sigma\^{}2), given a sample X}
\CommentTok{\# (Check lecture slides for the formula)}
\NormalTok{llik }\OtherTok{\textless{}{-}} \ControlFlowTok{function}\NormalTok{(par, X) \{}
\NormalTok{  mu     }\OtherTok{\textless{}{-}}\NormalTok{ par[}\DecValTok{1}\NormalTok{]}
\NormalTok{  sigma2 }\OtherTok{\textless{}{-}}\NormalTok{ par[}\DecValTok{2}\NormalTok{]}
\NormalTok{  n      }\OtherTok{\textless{}{-}} \FunctionTok{length}\NormalTok{(X)}
  \ControlFlowTok{if}\NormalTok{ (sigma2 }\SpecialCharTok{\textless{}=} \DecValTok{0}\NormalTok{) }\FunctionTok{return}\NormalTok{(}\ConstantTok{Inf}\NormalTok{)  }\CommentTok{\# variance must be positive}
\NormalTok{  llik }\OtherTok{\textless{}{-}} \SpecialCharTok{{-}}\NormalTok{n }\SpecialCharTok{*} \FunctionTok{log}\NormalTok{(}\FunctionTok{sqrt}\NormalTok{(}\DecValTok{2} \SpecialCharTok{*}\NormalTok{ pi }\SpecialCharTok{*}\NormalTok{ sigma2)) }\SpecialCharTok{{-}} \FloatTok{0.5} \SpecialCharTok{*} \FunctionTok{sum}\NormalTok{((X }\SpecialCharTok{{-}}\NormalTok{ mu)}\SpecialCharTok{\^{}}\DecValTok{2} \SpecialCharTok{/}\NormalTok{ sigma2)}
  \FunctionTok{return}\NormalTok{(llik)}
\NormalTok{\}}

\CommentTok{\# Initial guesses for mu and sigma\^{}2 }
\CommentTok{\# (any finite Real values of mu and of sigma2 \textgreater{} 0 should work)}
\NormalTok{init }\OtherTok{\textless{}{-}} \FunctionTok{c}\NormalTok{(}\AttributeTok{mu =} \DecValTok{0}\NormalTok{, }\AttributeTok{sigma2 =} \DecValTok{1}\NormalTok{) }

\CommentTok{\# Run optimization}
\NormalTok{fit }\OtherTok{\textless{}{-}} \FunctionTok{optim}\NormalTok{(}\AttributeTok{par =}\NormalTok{ init, }\AttributeTok{fn =}\NormalTok{ llik, }\AttributeTok{X =}\NormalTok{ x, }
             \AttributeTok{method =} \StringTok{"L{-}BFGS{-}B"}\NormalTok{,         }\CommentTok{\# optimisation method to use}
             \AttributeTok{lower =} \FunctionTok{c}\NormalTok{(}\SpecialCharTok{{-}}\ConstantTok{Inf}\NormalTok{, }\FloatTok{1e{-}9}\NormalTok{),       }\CommentTok{\# Enforce positive variance (minimal allowed value: 10\^{}{-}9)}
             \AttributeTok{control =} \FunctionTok{list}\NormalTok{(}\AttributeTok{fnscale =} \SpecialCharTok{{-}}\DecValTok{1}\NormalTok{) }\CommentTok{\# To make it a maximisation problem}
\NormalTok{             )}

\NormalTok{fit}\SpecialCharTok{$}\NormalTok{par}
\end{Highlighting}
\end{Shaded}

\begin{verbatim}
##        mu    sigma2 
## 11.993268  9.063433
\end{verbatim}

\hypertarget{q4-numerical-computation-of-mle-value}{%
\subsubsection{Q4: Numerical computation of MLE
value}\label{q4-numerical-computation-of-mle-value}}

\begin{Shaded}
\begin{Highlighting}[]
\DocumentationTok{\#\# Question 4.a}
\FunctionTok{set.seed}\NormalTok{(MY\_STUDENT\_ID) }\CommentTok{\# \textless{}{-}{-}{-} Don\textquotesingle{}t change this}

\CommentTok{\# Your code here}
\DocumentationTok{\#\#}
\end{Highlighting}
\end{Shaded}

\begin{Shaded}
\begin{Highlighting}[]
\DocumentationTok{\#\# Question 4.b}
\FunctionTok{set.seed}\NormalTok{(MY\_STUDENT\_ID) }\CommentTok{\# \textless{}{-}{-}{-} Don\textquotesingle{}t change this}

\CommentTok{\# Your code here}
\DocumentationTok{\#\#}
\end{Highlighting}
\end{Shaded}

\begin{center}\rule{0.5\linewidth}{0.5pt}\end{center}

\hypertarget{part-ii-data-visualisation}{%
\subsection{Part II: Data
visualisation}\label{part-ii-data-visualisation}}

\hypertarget{q5-basic-plotting}{%
\subsubsection{Q5: Basic plotting}\label{q5-basic-plotting}}

\begin{Shaded}
\begin{Highlighting}[]
\DocumentationTok{\#\# Question 5.a}
\FunctionTok{set.seed}\NormalTok{(MY\_STUDENT\_ID) }\CommentTok{\# \textless{}{-}{-}{-} Don\textquotesingle{}t change this}

\CommentTok{\# Your code here}
\DocumentationTok{\#\#}
\end{Highlighting}
\end{Shaded}

\begin{Shaded}
\begin{Highlighting}[]
\DocumentationTok{\#\# Question 5.b}
\FunctionTok{set.seed}\NormalTok{(MY\_STUDENT\_ID) }\CommentTok{\# \textless{}{-}{-}{-} Don\textquotesingle{}t change this}

\CommentTok{\# Your code here}
\DocumentationTok{\#\#}
\end{Highlighting}
\end{Shaded}

\begin{Shaded}
\begin{Highlighting}[]
\DocumentationTok{\#\# Question 5.c}
\FunctionTok{set.seed}\NormalTok{(MY\_STUDENT\_ID) }\CommentTok{\# \textless{}{-}{-}{-} Don\textquotesingle{}t change this}

\CommentTok{\# Your code here}
\DocumentationTok{\#\#}
\end{Highlighting}
\end{Shaded}

\begin{Shaded}
\begin{Highlighting}[]
\DocumentationTok{\#\# Question 5.d}
\FunctionTok{set.seed}\NormalTok{(MY\_STUDENT\_ID) }\CommentTok{\# \textless{}{-}{-}{-} Don\textquotesingle{}t change this}

\CommentTok{\# Your code here}
\DocumentationTok{\#\#}
\end{Highlighting}
\end{Shaded}

\hypertarget{q6-facetting-and-annotation}{%
\subsubsection{Q6: Facetting and
annotation}\label{q6-facetting-and-annotation}}

\begin{Shaded}
\begin{Highlighting}[]
\DocumentationTok{\#\# Question 6.a}
\FunctionTok{set.seed}\NormalTok{(MY\_STUDENT\_ID) }\CommentTok{\# \textless{}{-}{-}{-} Don\textquotesingle{}t change this}

\CommentTok{\# Your code here}
\DocumentationTok{\#\#}
\end{Highlighting}
\end{Shaded}

\begin{Shaded}
\begin{Highlighting}[]
\DocumentationTok{\#\# Question 6.b}
\FunctionTok{set.seed}\NormalTok{(MY\_STUDENT\_ID) }\CommentTok{\# \textless{}{-}{-}{-} Don\textquotesingle{}t change this}

\CommentTok{\# Your code here}
\DocumentationTok{\#\#}
\end{Highlighting}
\end{Shaded}

\begin{Shaded}
\begin{Highlighting}[]
\DocumentationTok{\#\# Question 6.c}
\FunctionTok{set.seed}\NormalTok{(MY\_STUDENT\_ID) }\CommentTok{\# \textless{}{-}{-}{-} Don\textquotesingle{}t change this}

\CommentTok{\# Your code here}
\DocumentationTok{\#\#}
\end{Highlighting}
\end{Shaded}



\end{document}
